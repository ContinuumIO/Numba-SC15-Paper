% THIS IS SIGPROC-SP.TEX - VERSION 3.1
% WORKS WITH V3.2SP OF ACM_PROC_ARTICLE-SP.CLS
% APRIL 2009
%
% It is an example file showing how to use the 'acm_proc_article-sp.cls' V3.2SP
% LaTeX2e document class file for Conference Proceedings submissions.
% ----------------------------------------------------------------------------------------------------------------
% This .tex file (and associated .cls V3.2SP) *DOES NOT* produce:
%       1) The Permission Statement
%       2) The Conference (location) Info information
%       3) The Copyright Line with ACM data
%       4) Page numbering
% ---------------------------------------------------------------------------------------------------------------
% It is an example which *does* use the .bib file (from which the .bbl file
% is produced).
% REMEMBER HOWEVER: After having produced the .bbl file,
% and prior to final submission,
% you need to 'insert'  your .bbl file into your source .tex file so as to provide
% ONE 'self-contained' source file.
%
% Questions regarding SIGS should be sent to
% Adrienne Griscti ---> griscti@acm.org
%
% Questions/suggestions regarding the guidelines, .tex and .cls files, etc. to
% Gerald Murray ---> murray@hq.acm.org
%
% For tracking purposes - this is V3.1SP - APRIL 2009

\documentclass{acm_proc_article-sp}

\begin{document}

\title{Numba: LLVM Practitioner Report}
% \subtitle{[Extended Abstract]
%\titlenote{A full version of this paper is available as
%\textit{Author's Guide to Preparing ACM SIG Proceedings Using
%\LaTeX$2_\epsilon$\ and BibTeX} at
%\texttt{www.acm.org/eaddress.htm}}}
%
% You need the command \numberofauthors to handle the 'placement
% and alignment' of the authors beneath the title.
%
% For aesthetic reasons, we recommend 'three authors at a time'
% i.e. three 'name/affiliation blocks' be placed beneath the title.
%
% NOTE: You are NOT restricted in how many 'rows' of
% "name/affiliations" may appear. We just ask that you restrict
% the number of 'columns' to three.
%
% Because of the available 'opening page real-estate'
% we ask you to refrain from putting more than six authors
% (two rows with three columns) beneath the article title.
% More than six makes the first-page appear very cluttered indeed.
%
% Use the \alignauthor commands to handle the names
% and affiliations for an 'aesthetic maximum' of six authors.
% Add names, affiliations, addresses for
% the seventh etc. author(s) as the argument for the
% \additionalauthors command.
% These 'additional authors' will be output/set for you
% without further effort on your part as the last section in
% the body of your article BEFORE References or any Appendices.

\numberofauthors{8} %  in this sample file, there are a *total*
% of EIGHT authors. SIX appear on the 'first-page' (for formatting
% reasons) and the remaining two appear in the \additionalauthors section.
%
\author{
% You can go ahead and credit any number of authors here,
% e.g. one 'row of three' or two rows (consisting of one row of three
% and a second row of one, two or three).
%
% The command \alignauthor (no curly braces needed) should
% precede each author name, affiliation/snail-mail address and
% e-mail address. Additionally, tag each line of
% affiliation/address with \affaddr, and tag the
% e-mail address with \email.
%
% 1st. author
\alignauthor
Siu Kwan Lam\\
       \affaddr{Continuum Analytics}\\
       \affaddr{Somewhere}\\
       \affaddr{Austin, Texas}\\
       \email{siu@continuum.io}
% 2nd. author
\alignauthor
Add Your Name\\
       \affaddr{Institute for Clarity in Documentation}\\
       \affaddr{P.O. Box 1212}\\
       \affaddr{Dublin, Ohio 43017-6221}\\
       \email{webmaster@marysville-ohio.com}
% 3rd. author
\alignauthor
Add Your Name\\
       \affaddr{The Th{\o}rv{\"a}ld Group}\\
       \affaddr{1 Th{\o}rv{\"a}ld Circle}\\
       \affaddr{Hekla, Iceland}\\
       \email{larst@affiliation.org}
\and  % use '\and' if you need 'another row' of author names
% 4th. author
\alignauthor Add Your Name\\
       \affaddr{Brookhaven Laboratories}\\
       \affaddr{Brookhaven National Lab}\\
       \affaddr{P.O. Box 5000}\\
       \email{lleipuner@researchlabs.org}
% 5th. author
\alignauthor Add Your Name\\
       \affaddr{NASA Ames Research Center}\\
       \affaddr{Moffett Field}\\
       \affaddr{California 94035}\\
       \email{fogartys@amesres.org}
% 6th. author
\alignauthor Add Your Name\\
       \affaddr{Palmer Research Laboratories}\\
       \affaddr{8600 Datapoint Drive}\\
       \affaddr{San Antonio, Texas 78229}\\
       \email{cpalmer@prl.com}
}
% There's nothing stopping you putting the seventh, eighth, etc.
% author on the opening page (as the 'third row') but we ask,
% for aesthetic reasons that you place these 'additional authors'
% in the \additional authors block, viz.
\additionalauthors{Additional authors: John Smith (The Th{\o}rv{\"a}ld Group,
email: {\texttt{jsmith@affiliation.org}}) and Julius P.~Kumquat
(The Kumquat Consortium, email: {\texttt{jpkumquat@consortium.net}}).}
\date{30 July 1999}
% Just remember to make sure that the TOTAL number of authors
% is the number that will appear on the first page PLUS the
% number that will appear in the \additionalauthors section.

\maketitle
\begin{abstract}
This is the abstract.
\end{abstract}

% A category with the (minimum) three required fields
\category{H.4}{TODO}{Miscellaneous}
%A category including the fourth, optional field follows...
\category{D.2.8}{TODO}{Metrics}[complexity measures, performance measures]

\terms{Theory}

\keywords{LLVM, Python, Compiler} % NOT required for Proceedings

\section{Background}

Python is a dynamically typed, interpreted language.
It is regarded as a high productivity language due to its simple syntax,
flexible semantic and the large number of libraries.
As a result, a scientific community have emerged and there is a stronger need
to better performance.
Among the many performance focused librarys, NumPy is one of the important
scientific library in the Python ecosystem as it provides a n-dimensional
array (ndarray) object that became the foundation of efficient numeric
computation in Python.

A ndarray is a homogenous typed data memory buffer.
Data can be stored in either C contiguous, FORTRAN contiguous or have arbitrary
strides at each dimension.  This allows binding to high performance computing
libraries that are traditionally written in C or FORTRAN, such as BLAS and
LAPACK. The ndarray support array expression: using Python operators on
ndarrays will trigger element-wise operations that are implemented in C
efficiently.  However, a trivial looping of the elements in Python is
inefficient due to the layers of indirection inside interpreted code.

\section{What is Numba?}

Numba is a just-in-time compiler for Python.
It uses LLVM for code generation.
Its initial focus is to target a Python subset that makes heavy use of
ndarrays in loops.  The ndarray provides important dimensionality,
type and data layout information that allows Numba to generate specialized
loops for arrays in machine code. By knowing the native structure of ndarray,
it bypasses unnecessary indirection for accessing data in ndarrays.
The basic structure of the ndarray consists of the data pointer to the base of
the memory buffer and two integer arrays for the dimensionality (aka shape) and
the strides. Numba can directly access these fields for calculating the offset
of each element.  As a result, it can generate efficient loops that indexes
into ndarrays that are as fast as their FORTRAN counterpart
(TODO: benchmark for proof).

(TODO: talk about current language support)

(TODO: talk about bypassing the CPython interpreter)

Unlike most JIT for interpreter languages, Numba does not perform tracing.
Instead, it requires user annotation for marking the function that requires
compilation. The just-in-time aspect comes from call-site code generation.
To relieve user from the burden of type annotation, Numba inspect the types
of the arguments for local type inference on the callee function.
As a result, it can have accurate type information for each values.
Python duck-typing is supported by overloading the function for each
encountered type combination at the call-site. (Talk about code bloat???)

Numba is not a perfect JIT as the semantic of the compiled code slightly differs
from the interpreted. It can compile a subset of the language only
and it violates some semantic of Python.  It does not support big integers
and is limited to 64-bit integer for the maximum representable integer type.
All arithmetic operations on 64-bit integer wraparound when the result
overflow.  Variable type are not polymorphic.  Implicit coercion occurs when
a type is assigned to a variable of a different type. In practice, these
restrictions and deviations from Python semantic rarely affect the user
because the ndarray has similar limitations. On the other hand, these
limitations allow for aggressive optimization.

A novelty of Numba is its multi-target backend.  It currently supports NVIDIA
CUDA backend by using NVVM.  Support for AMD HSA is also available on APU by
using HLC.  Both NVVM and HLC are vendor-specific version of LLVM for
additional support for their hardware.  Multi-target support is a important
reason for the restricted semantic.  Both CUDA and HSA are GPGPU architectures.
Highly divergent code runs inefficiently on GPGPUs due to thread divergence.
Executing certain GPGPU specific constructs, like the barrier, by diverged
threads results in undefined behavior. Forcing each variable to be single-typed
avoids implicitly branching that are hidden from user for fine-tuning the
performance and to avoid undefined behavior.


\section{Other Python JIT}

(TODO: Unlaiden Swallon, PyPy, Pyston)

\section{Numba Optimize}


\subsection{Type erasure}

(TODO: Numba lowered code does not contain type information)

\subsection{Generating Optimized Code for Ndarray}

(TODO: Convey buffer contiguousness by GEP and not strides.
NumPy uses strides only.)


\section{Managing LLVM API}

LLVM development is rapid and its C++ API changes frequently.
In early Numba development, we maintained a python binding, called llvmpy,
to the C++ API and have tried to keep a binding that mirrors the C++ interface,
but this was soon proven to be difficult.
To simplify the binding and to minimize the exposure the to the LLVM C++ API,
we adopted the C API instead and added any missing interface to the C++ API.
The new binding is called llvmlite.  It replicates the C++ IRBuilder API in
pure Python and builds up a string of the LLVM IR.

(TODO: talk about C assertion in LLVM)


\section{Challenges due to LLVM}
Subsection

\subsection{ABI is function type, portability}

TODO

\subsection{Slow code emission, OrcJIT}

TODO

\subsection{Cross module linkage, OrcJIT}

TODO

\subsection{Object Cache}

LLVM's MCJIT, which is used by Numba, features callback hooks for
applications to implement a form of object caching.  It works
as follows (*).  Before compiling a module, the MCJIT execution engine
calls the user-provided getObject() method.  That method can either
return a pointer to an existing memory area, in which case that
memory area is treated as an object file containing the module's
generated code (for example a ELF file under Linux), or a null pointer,
in which case compilation goes on as usual.

(*) API doc at http://llvm.org/docs/doxygen/html/classllvm_1_1ObjectCache.html

Then, whenever the MCJIT execution engine has finished compiling a
module (specifically, this means the getObject() method has returned
a null pointer), it calls the notifyObjectCompiled() method providing
both the module reference and a pointer to the memory area containing
the module's object code.

This callback-based model works well for a trivial form of caching,
where an LLVM module's object code is cached as-is, without any
additional information (this is how object caching is demonstrated in
the official Kaleidoscope tutorial (**)).  However, more complex forms of
caching need to serialize additional information beside the object code
for the module.  In Numba, this ancillary information is two-fold.  First,
we need to store freshness information (a timestamp, a version number,
and potentially other data) in order to decide at runtime whether the
cached module is fresh or stale.  Second, we also need to store non-LLVM
data that is necessary to call the module's functions to be called,
information which is a akin to a closure and is composed of arbitrarily
complex Python objects.

Both pieces of information are not known at the LLVM module layer;
they reside at a higher level of abstraction.  We therefore had to
turn the callback model upside-down by keeping the object code inside
some long-lived variables, in order to be able to save or load
the object code at the place and time where it is natural to do so.

For our use case, and presumably for other non-trivial JIT use cases,
it would have been better to rely on a simple imperative API, allowing
both to fetch a module's object code (when compiled), and to feed object
code to the execution engine for a module.

A callback-based API is essentially useful when events are produced from
the outside, and/or in an uncontrolled way: for example, GUI or network
events.  Here, the events are predictable by the application
itself (compilation is triggered deterministically by calls to the
execution engine API), therefore an imperative API would have been as
well suited to the task.

(**) http://blog.llvm.org/2013/08/object-caching-with-kaleidoscope.html


\section{Conclusions}
My conclusion
%\end{document}  % This is where a 'short' article might terminate

%ACKNOWLEDGMENTS are optional
\section{Acknowledgments}
Optional section

%
% The following two commands are all you need in the
% initial runs of your .tex file to
% produce the bibliography for the citations in your paper.
\bibliographystyle{abbrv}
\bibliography{sigproc}  % sigproc.bib is the name of the Bibliography in this case
% You must have a proper ".bib" file
%  and remember to run:
% latex bibtex latex latex
% to resolve all references
%
% ACM needs 'a single self-contained file'!
%


% That's all folks!
\end{document}
